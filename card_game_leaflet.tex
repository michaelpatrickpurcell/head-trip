\documentclass[a4paper, 10pt,notumble]{leaflet}
%Adjust paragraph indentation.
\setlength{\parindent}{0pt}

% Diagram for cover art
\usepackage{tikz}

% Adjust formatting of list environments
\usepackage{enumitem}
\usepackage{calc}

\setlist[itemize]{noitemsep, topsep=0pt, leftmargin=0.75cm}

\setlist[enumerate]{noitemsep, topsep=0pt, leftmargin=0.75cm}

\setlist[description]{noitemsep, topsep=0pt, labelindent=0.25cm, leftmargin=\widthof{\hspace{0.25cm}\textbullet\space}, font=\normalfont\textbullet\bfseries\space}

\newlist{playerlist}{description}{1}

\setlist[playerlist]{noitemsep, topsep=0pt, labelindent=0.25cm, leftmargin=\widthof{\hspace{0.25cm}},, labelwidth=\widthof{\ace\diamonds\ (Student) -}, labelsep=\widthof{\space}, font=\normalfont}

% Setting Fonts
\usepackage{fontspec}
%\setmainfont{XCharter}
\renewfontfamily\sectfont{Clarendon-Bold}

% Horrible hack to get small caps to work with Clarendon-Bold font.
\newcommand\fauxsc[1]{\fauxschelper#1 \relax\relax}
\def\fauxschelper#1 #2\relax{%
  \fauxschelphelp#1\relax\relax%
  \if\relax#2\relax\else\ \fauxschelper#2\relax\fi%
}
\def\fauxschelphelp#1#2\relax{%
  \ifnum`#1=\lccode`#1\relax\large{\char\uccode`#1}\else%
    \Large{#1}\fi%
  \ifx\relax#2\relax\else\fauxschelphelp#2\relax\fi}

% Macros for symbols that appear on playing cards
\newfontfamily\cardfont{Card Characters}[Scale=1.0]

\DeclareRobustCommand\spades[1][black]{\textcolor{#1}{\cardfont{\}}}}
\DeclareRobustCommand\hearts[1][red]{\textcolor{#1}{{\cardfont{\{}}}}
\DeclareRobustCommand\diamonds[1][red]{\textcolor{#1}{{\cardfont{[}}}}
\DeclareRobustCommand\clubs[1][black]{\textcolor{#1}{\cardfont{]}}}

\DeclareRobustCommand\two[1][black]{\textcolor{#1}{\cardfont{2}}}
\DeclareRobustCommand\three[1][black]{\textcolor{#1}{\cardfont{3}}}
\DeclareRobustCommand\four[1][black]{\textcolor{#1}{\cardfont{4}}}
\DeclareRobustCommand\five[1][black]{\textcolor{#1}{\cardfont{5}}}
\DeclareRobustCommand\six[1][black]{\textcolor{#1}{\cardfont{6}}}
\DeclareRobustCommand\seven[1][black]{\textcolor{#1}{\cardfont{7}}}
\DeclareRobustCommand\eight[1][black]{\textcolor{#1}{\cardfont{8}}}
\DeclareRobustCommand\nine[1][black]{\textcolor{#1}{\cardfont{9}}}
\DeclareRobustCommand\ten[1][black]{\textcolor{#1}{\cardfont{=}}}

\DeclareRobustCommand\jack[1][black]{\textcolor{#1}{\cardfont{J}}}
\DeclareRobustCommand\queen[1][black]{\textcolor{#1}{\cardfont{Q}}}
\DeclareRobustCommand\king[1][black]{\textcolor{#1}{\cardfont{K}}}
\DeclareRobustCommand\ace[1][black]{\textcolor{#1}{\cardfont{A}}}
\DeclareRobustCommand\joker[1][black]{\textcolor{#1}{\cardfont{\textcolor{red}{J}\textcolor{black}{O}\textcolor{red}{K}\textcolor{black}{E}\textcolor{red}{R}}}}

\DeclareRobustCommand\redqueen[1][red]{\textcolor{#1}{\cardfont{Q}}}
\DeclareRobustCommand\redace[1][red]{\textcolor{#1}{\cardfont{A}}}

% Start Document

\begin{document}

% Cover page with title, cover art, description, and author

\setmainfont[Scale=2.8]{Clarendon-Bold}
\begin{center}
\begin{tabular}{c@{\hspace{0.25ex}}l@{\hspace{0.05ex}}c@{\hspace{0.25ex}}l}
\huge{H} & \huge{\textcolor{red}{E}} & \huge{A} & \huge{\textcolor{red}{D}} \\[0.5ex]
\huge{\textcolor{red}{T}} & \huge{R} & \huge{\textcolor{red}{I}} & \huge{P}
\end{tabular}
\end{center}

\vfill

\begin{figure}[h]
\centering
	\begin{tikzpicture}
	\node (A) at (0,0) {\includegraphics[scale=0.06]{head_outline_small.png}};
	\node (B) at (1.1,0.95) {\fontsize{60}{72}{\hearts}};
	\node (B) at (-1.9,0.95) {\fontsize{60}{72}{\diamonds}};
	\node (B) at (-0.475,2.6) {\fontsize{60}{72}{\spades}};
	\node (B) at (-0.475,-0.4) {\fontsize{60}{72}{\clubs}};
	\end{tikzpicture}
\end{figure}

\vfill

\setmainfont[Scale=1.5]{XCharter}
\begin{center}
A card game for four players

\smallskip

designed by Michael Purcell
\end{center}

\newpage

\setmainfont{XCharter}
\raggedright
\section*{\fauxsc{Typesetting Tests}}
These are card suits: \spades, \hearts, \diamonds, \clubs.

These are number-card symbols:
\begin{itemize}
	\item \textbf{Card Font}: \two, \three, \four, \five, \six, \seven, \eight, \nine, \ten.
	\item\textbf{Body Font}: 2, 3, 4, 5, 6, 7, 8, 9, 10.
\end{itemize}

These are face-card symbols:
\begin{itemize}
	\item \textbf{Card font}: \jack, \queen, \king, \ace.
	\item \textbf{Body font}: J, Q, K, A.
\end{itemize}

These are mixed symbols:
\begin{itemize}
	\item \textbf{Card Font}: \ace\spades, \ten[red]\hearts, \queen[red]\diamonds, \seven\clubs.
\end{itemize}

Font tests: This is a test of regular-weight text, \textbf{bold-weight text}, \textit{italic-shape text}, \textbf{\textit{bold-weight and italic-shape text}}, \textsc{Small-Caps shaped text}.

This is a test of \emph{emphasis}.

\newpage

Page 3

\newpage

\section*{\fauxsc{Game Play}}
The game takes place over a sequence of rounds.
During each round:
\begin{enumerate}
	\item The players must reveal the top card of the challenge deck. 
	\item The pilot must face the challenge or stand down.
	\item If the pilot chooses to face the challenge, the other players may play a \joker\ to force the pilot to stand down instead.
	\item If there is no pilot, the players must elect one.%player to assume the role.
	\item The pilot must face the current challenge. 
\end{enumerate}
The game ends when any player plays their last card or after the pilot faces the last challenge.

\subsection{Elections}
To elect a pilot, the players must each select an aspect card to indicate who they are voting for. 
The player who receives the most votes wins the election. 

In the case of a tie, \ace[red]\diamonds (Student) must decide which of the leading vote-getters wins the election. 

The winner of the election must become the new pilot.

\subsection{Facing Challenges}
To face a challenge, the pilot must play a card to the tableau.
If the face-value of the pilot's card meets or exceeds the face-value of the challenge card, then the pilot succeeds. Otherwise, the pilot fails.

The pilot's card is added to the tableau in the row corresponding to the suit of the challenge card. If the pilot succeeds, then their card is played face up.  Otherwise, their card is played face down. 

\subsection{Scoring}
The score for each player is given by:
\begin{playerlist}
	\item[\jack\clubs\hfill(Animal) -] total \clubs\ (physical) successes.
	\item[\redqueen\hearts\hfill(Angel) -] total \hearts[red]\ (social) successes.
	\item[\king\spades\hfill(Genius) -] total \spades\ (mental) successes.
	\item[\redace\diamonds\hfill(Student) -] total failures.
\end{playerlist}
The players with the most points win the game.
\newpage

\section{\fauxsc{Optional Rules}}
Test

\newpage

\section{\fauxsc{Acknowledgements}}
Test

\end{document}