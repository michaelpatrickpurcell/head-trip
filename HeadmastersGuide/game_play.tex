\section{Game Play}
The game takes place over a sequence of rounds. During each round, the Headmaster should:
\begin{enumerate}
	\item Establish the setting for the current challenge. 
	\item Choose a challenge card for the pilot to reveal. The  other players will then elect a pilot. That pilot will then play an action card to the tableau.
	\item React to the action card played by the pilot.
\end{enumerate}

\subsection{Establishing the Setting}
The Headmaster should describe the location where the current challenge will take place and any non-player characters who might be involved.

The location cards and character cards can be used as a menu of possible setting elements that the Headmaster can choose from or as a way to generate random setting elements.

Each setting should present Bobby with a situation where the outcome  of the challenge is uncertain and where Bobby stands to benefit if he succeeds. 


\subsection{Choosing a Challenge Card}
The Headmaster should choose a challenge card appropriate for the setting of each challenge.

Challenge cards should not be reused.  Once Bobby has faced a challenge represented by a given challenge card, that card is unavailable for use in future challenges and should be discarded.
  

\subsection{Reacting to an Action Card}
The Headmaster should work with the pilot to describe how Bobby faces each challenge.

This description should incorporate the type and difficulty of the challenge, the rank of the pilot's action card, and whether Bobby succeeds or fails.

Bobby is always the one who is facing each challenge, not the pilot. The difference between the aspects is their motivations rather than their capabilities.

