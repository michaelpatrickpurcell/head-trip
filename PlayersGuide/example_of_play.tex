\section{Example of Play}
The following is a description of a single round of play. This is the ninth round of the game (so eight cards have already been played to the tableau) and the~Angel is the current pilot.

\begin{enumerate}
	\item The Angel reveals the next challenge card~is~\four[red]\hearts.
	\item The Angel decides not to stand down. The Genius and the Animal object, and agree to spend a recall card to force her to step down.
	\item The players cast their votes for which aspect should become the next pilot. Their votes are:

	\begin{table}[h]
	\centering
	\begin{tabular}{l r}
	\toprule
	Aspect & Vote \\ \midrule
	The Animal & \jack\clubs \\
	The Angel  & \queen[red]\hearts \\
	The Genius & \jack\spades \\
	The Student & \jack[red]\diamonds \\ \bottomrule
	\end{tabular}
	\end{table}
	
	That is, the Animal receives three votes and the~Angel receives one vote. So, the Animal will be the next pilot.
	\item The Animal decides to play \five\clubs, which he adds face up to the middle row of the tableau. This yields the partially-completed tableau:
	
\begin{figure}[h]\centering
\begin{tikzpicture}[scale=0.9]
	\pic[rotate=90, transform shape] () at (0,32mm) {clubscard={\ace}};
	\pic () at (15mm, 32mm) {diamondscard={\six[red]}};
	\pic () at (27mm, 32mm) {spadescard={\eight}};
	\pic () at (39mm, 32mm) {cardback};

	\pic[rotate=90, transform shape] () at (0,16mm) {heartscard={\redace}};
	\pic () at (15mm, 16mm) {cardback};
	\pic () at (27mm, 16mm) {heartscard={\seven[red]}};
	\pic () at (39mm, 16mm) {heartscard={\five[red]}};
	\pic () at (51mm, 16mm) {clubscard={\five}};

	\pic[rotate=90, transform shape] () at (0,0) {spadescard={\ace}};
	\pic () at (15mm, 0) {clubscard={\eight}};
	\pic () at (27mm, 0) {cardback};
\end{tikzpicture}
\caption*{A partially-completed tableau.}
\end{figure}

\bigskip
After this round the Student and the~Angel both have three points, the Animal has two points, and the Genius has one point.

\end{enumerate}

