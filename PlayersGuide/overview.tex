\section{Overview}
Head Trip is a game for four players that is played with two standard decks of playing cards. Each player will portray an \emph{aspect} of the personality of a teenage boy named Bobby: 
\begin{playerlist}
	\item[\jack\clubs\ (The Animal):] represents Bobby's physicality.
	\item[\redqueen\hearts\ (The Angel):] represents Bobby's compassion.
	\item[\king\spades\ (The Genius):] represents Bobby's intellect.
	\item[\redace\diamonds\ (The Student):] represents Bobby's curiosity.
\end{playerlist}

During the game, Bobby will face a series of challenges.
There are three types of challenges: \clubs~(physical), \hearts~(social), and \spades~(mental). Different aspects receive different rewards depending on how Bobby responds to each challenge.

The player whose aspect receives the most rewards as result of Bobby's actions will win the game.

However, only one aspect can control Bobby's actions at any given time. This aspect is called the \emph{pilot}. So, the aspects will have to compete for control to ensure that they Bobby does what they want and that they receive as many rewards as possible.

\section{The Cards}
The \emph{persona deck} is a complete (54 cards) standard deck of playing cards. It consists of:
\begin{personadecklist}
	\item[Action Cards\hfill\normalfont{(36):}] all \two\ \textendash\ \ten\ cards.
	\item[Voting Cards\hfill\normalfont{(16):}] all \jack, \queen, \king, \ace\ cards.
	\item[Recall Cards\hfill\normalfont{(2):}] all \joker\ (joker) cards.
\end{personadecklist}

The \emph{auxiliary deck} is a subset (34 cards) of a standard deck of playing cards. It consists of:
\begin{auxiliarydecklist}
	\item[Challenge Cards\hfill\normalfont{(27):}] \two\ \textendash\ \ten\ of \clubs, \hearts, \spades.
	\item[Aspect Cards\hfill\normalfont{(4):}] \jack\clubs, \queen[red]\hearts[red], \king\spades, \ace[red]\diamonds[red].
	\item[Label Cards\hfill\normalfont{(3):}] \ace\clubs, \ace[red]\hearts, \ace\spades.
\end{auxiliarydecklist}

It may be useful to use decks with different coloured backs or decks of different sizes to make it easier to see which cards belong to each deck.

